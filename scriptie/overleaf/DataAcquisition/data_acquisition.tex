\outline{
\begin{itemize}
    \item Onderzoeksvraag: Welke data is nodig en hoe kom je eraan?
    \item Doelstellingen:
    \begin{itemize}
        \item Data verzamelen
        \item Data analyseren
    \end{itemize}
\end{itemize}
}



\chapter{Data models (hele slechte titel)}
\outline{
\begin{itemize}
    \item Model wat in dblp, aminer en in HJ staat
    \item Model publicatieprocess
    \item Sluit niet aan
    \item meer rollen -> niet in publieke datasets
\end{itemize}
}

\chapter{Data Acquisition}

\outline{
\begin{itemize}
    \item Publicly available sources are limited in available data. Most datasets provides information on who wrote what and sometimes which article refers what. This provides a good insight and based on these datasets a lot of research is already conducted.
    \item However, we think these datasets miss actors on the publication process that are in a position with enough power to influence the publication process.
    \item Data about these people involved in the publication process are not structured available to use in analysis. The information is provided in front matters and masthead documents. Because of the nature of these documents, mostly PDF's, we consider this as unstructured data.
    \item We are somehow surprised of the lack of structured data availability about these people, because of their power and influence.
    \item Our focus in this chapter is to incorporate this less structured available data into publicly available data about the publication process to get a dataset which provides a more complete view of the publication process.
    \item Our research focus on information technology so our start point to gather these people is to use the most prominent conferences and journals. As source for this data we use the core ranking.
\end{itemize}

\outline{
\section{Problem statement}
\subsection{Current available data}
\begin{itemize}
    \item Most public available datasources for the publicationprocess contain information about the author and what the author publices. Sometimes (but not all, or incomplete) these datasets also contain references.
    \item The model of these datasets are visualized by the publication process of Jonker and Mauw.
    \item Examples of these datasets are DBLP and Aminer.
\end{itemize}
\subsection{Model of publicationprocess}
\subsection{Mismatch}
\begin{itemize}
    \item Tielemans vooral gebasseerd op beschikbare data (DBLP)
    \item Machtsverhoudingen komen niet terug in de publieke datasets
\end{itemize}
\subsection{Justification}
\begin{itemize}
    \item Overzicht real-life cases (probleem verder beschrijven)
    \begin{itemize}
        \item Voorbeeld Vasilakos, citaten uit 2 journals waar hij zelf in zit
        \item Publicatie lag
    \end{itemize}

\end{itemize}
}

\section{Our root: Computing Research and Education Association of Australasia ranking}
\begin{itemize}
    \item CORE
    \item Australian based, so might be biased
    \item A website which ranks conferences and journals
    \item Ranking is from A* to C. Our focus is for the ranking A*, A and B
    \item The conference list contains the following attributes:
    \begin{itemize}
        \item Title
        \item Acronym
        \item Rank
        \item DBLP link
    \end{itemize}
\end{itemize}
\subsection{Gathering the data}
\begin{itemize}
    \item Choices for data acquisition:
    \begin{itemize}
        \item Use the export functionality on the website
        \item Scrape the data from the HTML pages
    \end{itemize}
    \subsection{Methods}
    \paragraph{Export functionality}
    The core websites offers the functionality to download all the data in a CSV file. This is very useful, because 1) this allows us to get the data in a format that is simple to process in a database (we can consider this a table) and 2) it is fast to get the data.
    
    The drawback however is that this data does not contain the DBLP url. This means we are not able to find related data in a structured manner.
    A possible way to work with this problem is to manually add the dblp url's ourselves, but this is a labour intensive job, comes with a cost in reproducability and, because of the manual effort, is error-prone.
    
    \paragraph{Webscraping}
    \begin{itemize}
        \item The core website contains the data in a HTML table accross multiple pages.
        \item With GET request able to get this -> so we are able to automate this.
        \item All data that is in the table is at our disposal.
        \item Also the dblp url in the href of the Anchor tag.
        \item Drawback: Need to build a script for it, which take more time than simply click the export button.
    \end{itemize}
    
    \paragraph{Copy-pasting from the website}
    \begin{itemize}
        \item Html table, so easy to paste in a table
        \item Manual work: not convenient for reproducability
        \item Also unable to find the DBLP url (we only get the 'view'-text in this case)
    \end{itemize}
    
    \paragraph{Method conclusion}
    \begin{itemize}
        \item We choose to build a script to scrape the data over downloading the export
        \item Main reason was reproducability and less error prone in finding the DBLP urls.
    \end{itemize}
    
    \subsection{Process}
    \begin{itemize}
        \item pagination is given in the url
        \item While testing the site we discovered that when requesting a page outside the pagination number, we did not get a 200 (OK) response
        \item Loop through pages untill we do not get a 200 (OK) response
        \item Get the data from the table
        \item Write it to database
        \item build in python because... not really a reason except it's hip and happening (and I honestly don't know why)
        
    \end{itemize}
    
    \subsection{Result}
    \item The result was the data presented in two tables in schema 'core': 1) conf-ranks (conference rankings) and jnl-ranks (journal rankings)
    \paragraph{Completeness}
    Core conference data:
    \begin{itemize}
        \item 883 rows in total, within scope (because of rank): 486
        \item No DBLP link for 24 conferences within our ranking focus.
        \item To get this data we manually looked up the dblp url and added this data to the database. We were able to to find the dblp urls of 17 conferences, which remains 7 unknown
    \end{itemize}
    
    
    \item From this point we split the data acquisition process in two lanes:
    \begin{itemize}
        \item Conference tracks
        \item Journal tracks
    \end{itemize}
    
    
\end{itemize}


\section{Conference track}
\begin{itemize}
    \item Input is conf-ranks form core combined with manual searched DBLP urls
    \item Purpose is to get the proceedings of the conferences
    \item One conferences has one or more proceedings
    \item DBLP contains this data
\end{itemize}

\subsection{Gathering DBLP data}
\paragraph{Methods}
\begin{itemize}
    \item Download DBLP database
    \begin{itemize}
        \item Already done in project preparation
        \item BUT; looking for certain proceedings were not available in download, but do on web
    \end{itemize}
    \item Use the API
    \begin{itemize}
        \item DBLP offers an API to download information.
    \end{itemize}
\end{itemize}
We used the API to extract the data. Because the data is more complete.

For all conferences in the core table (combined with some manual adjustment), we extracted the proceedings from DBLP.
This resulted in 15994 proceedings.

Looking for the most used publisers:
\begin{table}[h]
\begin{tabular}{ll}
\hline
Publisher & \# of proceedings \\ \hline
Springer  & 4674              \\
ACM       & 4010              \\
IEEE *    & 3592             
\end{tabular}
\end{table}
*IEEE also includes IEEE Computer Society
} % END OUTLINE

\section{PDF Reading}
\begin{itemize}
    \item PDF is mostly text plotted on a sheet with certain coordinates and properties. (also images, but we are not interested in these)
    \item options: PDF to html
    \begin{itemize}
        \item Problem changed from interpreting PDF to interpreting HTML
        \item Still need to be able to 'understand' the relationships between certain elements on the page
    \end{itemize}
    \item OCR (tesseract)
    \begin{itemize}
        \item OCR is a technology to detect characters from images
        \item PDF is not an image, so first need to convert the PDF's to images
        \item Output is plain text or TSV. Lot of contect (like position) gets lost in the process. This information is crucial if we want to interpret the pages.
    \end{itemize}
\end{itemize}
\subsection{pdf to html}
\begin{itemize}
    \item Volgende libraries geprobeerd:
    \begin{itemize}
        \item PDFMiner
        \begin{itemize}
            \item Tries to split blocks of text
            \item can be nice, but in these kind of documents, there is mostly a relation between the columne (like 'person' and an 'affiliation')
        \end{itemize}
        \item pdf2htmlex
        \begin{itemize}
            \item Most promising
            \item Lots of CSS
            \item Spaces can be tricky. Sometimes there is a space, but with a margin class with a negative number, so letter-space minus the margin, becomes so little, we have to ignore this space.
            \item on the other hand, sometimes there is no space, but the space needs to be derived from the letter-spacing.
            \item other challenge are items in a column that do not fit the row, so it continues on the next row. How do we consider this text as part of which column?
        \end{itemize}
    \end{itemize}
\end{itemize}
\subsection{Interpreting HTML}
\begin{itemize}
    \item what is the scope of the document we need? For LNCS, Springer provides a template where this section is calles 'Organization'. This is not used in all cases. From whcih point in time did they choose to use this template? Are there other keywords we can use? (like of a lot of subsection headers contain words like 'committee' or 'chair').
    \item Can we set the scope by interpreting the format? Most of the lists with names are columns.
    \item When do we consider an element an affiliation? it seems that most (if not all) affiliation have a comma to wplit the country name.
    \item Can we consider all elements in a column if most of them have a comma as affiliation? No, sometimes names are also <lastname>, <firstname>. But when this is the case, this happens at the first column also, and affiliatons are not likely to be placed in the first column.
\end{itemize}
Approach 1: Directly raed HTML elements and see if we can make this work for the LNCS frontmatters.
\begin{itemize}
    \item Not sustainable
    \item hard to maintain
    \item lots of if-then rules
    \item good first step to see what is working and what we need to take into account
\end{itemize}
Approach 2: First create a model of the HTML document, from this model create a graph with sectionheaders and sections.
\begin{itemize}
    \item How to detect section headers? Option; get the font-size that is most used. consider this as the size of the content. Everything that is bigger, assume headers.
\end{itemize}

\chapter{Data extraction}
\outline{
\begin{itemize}
    \item Uitdiepen complexheid PDF extractie
\end{itemize}

