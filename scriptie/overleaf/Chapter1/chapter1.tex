
\newcommand{\mi}[1]{\ensuremath{\mathit{#1}}}
\newcommand{\authors}{\mi{authors}}
\newcommand{\cites}{\mi{cites}}
\newcommand{\receives}{\mi{receives}}
\newcommand{\reviews}{\mi{reviews}}
\newcommand{\accepts}{\mi{accepts}}
\newcommand{\rejects}{\mi{rejects}}
\newcommand{\editorinchief}{\mi{EiC}}
\newcommand{\associateeditor}{\mi{AE}}
\newcommand{\Humans}{\mi{Humans}}
\newcommand{\Reviewers}{\mi{Reviewers}}
\newcommand{\Editors}{\mi{Editors}}



\chapter{Publication process}
\outline{
\begin{itemize}
    \item Voor BH and ALI description; not images
    \item Flow editorial process (maybe draw)
\end{itemize}


\section{Roles in the editorial process}
In the editorial process multiple roles are involved. These roles are executed
by persons. We define all persons involved somehow in the editorial process by 
\textit{H}. Publications are defined by the letter \textit{P}.
% ------------------------------------------------------------------------------
\paragraph{Author}
\begin{itemize}
    \item Writes publication
    \item Suggest Associate Editor
    \item Suggest Reviewer
\end{itemize}
We define the collection authors of a paper $A(p) = \{h \in H \mid \authors(p, h)\}$

% ------------------------------------------------------------------------------
\paragraph{Editor-in-chief}
Wie bepaalt wie de editor-in-chief is? -> macht
\begin{itemize}
    \item receives the manuscript $\receives(p, h)$
    \item Performs initial check (relevance, suitability to undergo peer review)
    \item Assigns Associate Editor based on Area of expertise and avoiding potential conflicts of interest between author and AE
    \item end decision if paper is being published: Stel h = Editor in chief en p = publication dan: $\accepts(p, h)$ of $\rejects(p, h)$
    \item An editor-in-chief has this role for a certain period (span multiple issues).
\end{itemize}


% ------------------------------------------------------------------------------
\paragraph{Editorial assistance}
\begin{itemize}
    \item Checks similarity to other publication (iThenticate)
    \item Verder geen invloed op process
\end{itemize}

% ------------------------------------------------------------------------------
\paragraph{Reviewer}
\begin{itemize}
    \item Reviews a paper
\end{itemize}
We define the collection reviewers of a paper $R(p) = \{h \in H | reviews(p, h)\}$
% ------------------------------------------------------------------------------
\paragraph{Subeditor}
\begin{itemize}
    \item Niet echt relevant
    \item Om tekst leesbaarder te maken
\end{itemize}
% ------------------------------------------------------------------------------
\paragraph{Associate editor}
\begin{itemize}
    \item Identify and Assigns reviewers
    \item administrative reject (desk reject): vaak al in eerder stadum eruit gehaald door eic of ea

    \item recommendation to editor in chief
    \item Following roles sometimes combined into the Associate Editor:
    \item Managing editor
    \begin{itemize}
        \item Checks journal standards, word length, use of internal reporting standard
        \item assigns editor
        \item assigns reviewers
    \end{itemize}
    \item Editor
    \begin{itemize}
        \item Initial check
    \end{itemize}
    \item This role is also for a period of time
\end{itemize}
% ------------------------------------------------------------------------------

\paragraph{Interesting cases}




\begin{itemize}
    \item Author and reviewer are the same person: $\{h \in \Humans \mid \authors(p, h) \land \reviews(p, h)\}$ (Moon)
    \item Editor-in-Chief is part of authors: $\{h \in \Humans \mid \authors(p, h) \land \receives(p, h)\}$. Logisch gevolg is dus dat $\accepts(p, h)$
    \item Citaat naar paper van iemand in het proces: stel q is de publicatie waar het om draait dan:
    $\{h \in \Editors \cup \Reviewers \mid \cites(q, p) \land \authors(h, p)\}$
    \item bovengemiddeld zelfcitaties
\end{itemize}

}


\section{Flow conference}
\outline{
\begin{itemize}
    \item wat is macht program committee vs program chair; 
    
    \item Conferentie, core ranking: https://www.core.edu.au/conference-portal
    \item Idee\"en egregious examples -> als deze situatie voorkomt, wil ik het vinden
\end{itemize}
}
\subsection{Steering committee}
\begin{itemize}
    \item deelt macht uit, kiest Program chair(s)
\end{itemize}
\subsection{General chair}
\subsection{Program chair}
\begin{itemize}
    \item bepaalt wie gaat reviewen, eindverantwoordelijk voor programma / tracks die erin voor moeten komen
\end{itemize}
\subsection{program committee}
\begin{itemize}
    \item Program committee; verantwoordelijk evalueren papers en discussie over samenstelling programma
\end{itemize}

\chapter{Title of the chapter}

\section{DBLP}
\outline{
\begin{itemize}
    \item Wat is DBLP
    \begin{itemize}
        \item DBLP is a dataset with information about publication in the computer science discipline.
        \item 
        \item Transformed to relational tables.
        \item Result is 35 tables.
    \end{itemize}
    \item Structure of delivery
    \begin{itemize}
        \item Datastructure of DBLP is XML.
        \item contains bibliographic records; like example
        \item These bilbiographic records have elements such as author, title, journal
    \end{itemize}
    \item Processing
    \begin{itemize}
        \item convert XML to relational tables
        \item Every bibliographic type in separate table
        \item Assign unique id
        \item Make separate table for elements with id of bibliographic record
        \item Attributes of elements are columns in table
        \item Add metadata to all data entries
    \end{itemize}
    \item Result
    \begin{itemize}
        \item 35 tables
        \item overzicht van tabellen en relaties
    \end{itemize}
\end{itemize}    
}


\lstset{language=XML}
\begin{lstlisting}[caption={DBLP bibliografic record example},label={lst:DblpExample}]
<article mdate="2019-10-25" key="tr/gte/TM-0332-11-90-165" publtype="informal">
<author>Frank Manola</author>
<author>Mark F. Hornick</author>
<author>Alejandro P. Buchmann</author>
<title>Object Data Model Facilities for Multimedia Data Types.</title>
<journal>GTE Laboratories Incorporated</journal>
<volume>TM-0332-11-90-165</volume>
<month>December</month>
<year>1990</year>
<url>db/journals/gtelab/index.html#TM-0332-11-90-165</url>
</article>
\end{lstlisting}
\begin{itemize}
    \item Bibliografische records zijn gedefinieerd door een element met subelementen. Deze subelementen bevatten waardes, en geen andere elementen.
    
    \item 
    \begin{itemize}
        \item zie author element in voorbeeld
        \item moeten manier vinden om hier mee om te gaan
        \item oplossing is id generereren voor elk 'parent' object.
        \item dit zijn objecten die direct onder de tag <dblp> vallen.
    \end{itemize}
   
    
    
    \item deze objecten aparte tabellen opslaan
    \item waarbij tabel naam de naam van het attribuut is (bijv. article)
     \item ander probleem geen zuivere xml: 
    \lstset{language=XML}
    \begin{lstlisting}[caption={XML fout},label={lst:DblpExample2}]
    <title>Graphs of Bounded Treewidth can be Canonized in AC<sup>1</sup>.</title>
    \end{lstlisting}
    
\end{itemize}



\chapter{Data integration}
\outline{
    \begin{itemize}
        \item Three main sets to intergrate: Humans, Venues, Publications
    \end{itemize}
    \section{Publication}
    \begin{itemize}
        \item Root is DBLP
        \item Key is DBLP key
        \item We need to integrate the publications from DBLP with Aminer
        \item DBLP is leading, so we need to match publications from dblp to aminer
        \item In two steps: First try to match on DOI, second, match on title
    \end{itemize}
    \subsection{Document Object Identifier}
    \begin{itemize}
        \item Wat is het? DOI is unique identifier of a document
        \item Hoe in DBLP?
            DBLP contains additional information about an entry in an ee tag.
            These contain links to doi.org, wikidata, etc.
            Sometimes an article has multiple doi.org links; this seems to be the case with the publisher ACM.
            Although the links are different, they result in the same page at ACM.
            for articles:
            number of doi, number of articles
            0 -> 322169
            1 -> 1094002
            2 -> 728
            3 -> 61
            most articles do have a link to doi.org in the ee
    \end{itemize} 
    \subsection{Enrichtment of DOI's}
    Some publication do not have an DOI in DBLP, but provide an additional link to arxiv site, or wikidata
    \paragraph{Arxiv}
    \begin{itemize}
        \item Arxiv sometimes has a DOI (but not always).
        \item We extact docuemnt information from the Arxiv API. As we were there, we also extracted the authors.
    \end{itemize}
    \paragraph{Wikidata}
}

\chapter{Recommendations}
\outline {
\begin{itemize}
    \item transparantie publicatieprocess (wie zijn de reviewers)
\end{itemize}
}